\documentclass[twocolumn,pre,nobalancelastpage]{revtex4}
\usepackage{graphicx}
\usepackage{fancyhdr}
\usepackage{amsmath}
\usepackage[usenames]{color}
\usepackage[ps2pdf,colorlinks=true,citecolor=blue,linkcolor=blue,urlcolor=blue,pagebackref=true]{hyperref}

\lhead{\small \textsf{Frequency Noise (ver \today)}}
\rhead{\small \textsf{Marohn @ Cornell University}}


\begin{document}
\title{Frequency Noise}
\author {John A. Marohn (\href{mailto:jam99@cornell.edu}{jam99@cornell.edu})}
\affiliation{Department of Chemistry and Chemical Biology \\ 150 Baker Laboratory \\ Cornell University \\ Ithaca, New York 14853-1301}
\date{\today}
\maketitle
\thispagestyle{fancy}

\section{Detection of Instantaneous Phase}
%=========================================

The cantilever signal is
\begin{equation}
\boxed{x(t) = \sqrt{2} \: x_{\text{rms}} \cos{(\omega_0 t + \phi)} + \delta x(t)} \label{Eq:x}
\end{equation}
where $x_{\text{rms}}$ is the cantilever root mean square amplitude, $\omega_0$ is the cantilever frequency, and $\phi$ is the cantilever phase.  Here $\delta x(t)$ is random noise which includes contributions from cantilever thermomechanical fluctuations as well as detector noise.

In order to detect the cantilever frequency we create a quadrature signal by taking the Hilbert transform of the cantilever signal.  This procedure gives
\begin{equation}
y(t) = \sqrt{2} \: x_{\text{rms}} \sin{(\omega_0 t + \phi)} + \delta y(t) \label{Eq:y}
\end{equation}
where $\delta y(t)$ is the Hilbert transform of $\delta x(t)$.  An expression for $\delta y(t)$ can be written down, but it is not instructive.  There is a simple relation, however, between $y$ and $x$ in the Fourier domain:
\begin{equation}
\widehat{\delta y}(f) = H(f) \: \widehat{\delta x}(f)
\end{equation}
where $\widehat{\delta x}(f)$ indicates the Fourier transform of $\delta x(t)$.  The function $H$ implements the Hilbert transform in Fourier space:
\begin{equation}
H(f) = \begin{cases}
+j & \text{if } f < 0 \\
0 & \text{if } f = 0 \\
-j & \text{if} f > 0
\end{cases}
\end{equation}
Since $H(f) H^{*}(f) = 1$ (except for the single point at $f=0$), it follows that $\delta y(t)$ has essentially the same power spectrum as $\delta x(t)$.

In our frequency-detection algorithm we measure the instantaneous phase of the cantilever using
\begin{equation}
\phi(t) = \arctan{(\frac{y(t)}{x(t)})} \label{Eq:phi_def}
\end{equation}
Substituting Eqs.~\ref{Eq:x} and \ref{Eq:y} into Eq.\ref{Eq:phi_def},
\begin{equation}
\phi(t) = \arctan{(\frac{\sqrt{2} \: x_{\text{rms}} \sin{(\omega_0 t + \phi)} + \delta y(t)}
 {\sqrt{2} \: x_{\text{rms}} \cos{(\omega_0 t + \phi)} + \delta x(t)})}
\end{equation}
Let us now, with the help of Mathematica, expand $\phi(t)$ in a Taylor series to first order in \emph{both} $\delta y(t)$ and $\delta x(t)$.   The result is
\begin{multline}
\phi(t) \approx \phi + \omega_0 t
- \frac{\delta x(t)}{\sqrt{2} \: x_{\text{rms}}} \sin{(\omega_0 t + \phi)} \\
+ \frac{\delta y(t)}{\sqrt{2} \: x_{\text{rms}}} \cos{(\omega_0 t + \phi)}
\end{multline}
We can extract the instantaneous frequency as the slope of the $\phi(t)$ versus $t$ line.  After subtracting away the best-fit line, we are left with phase noise
\begin{equation}
\delta \phi(t) = \phi(t) - \omega_0 t - \phi
\end{equation}
given by
\begin{multline}
\color{Blue} \delta \phi(t) = - \frac{\delta x(t)}{\sqrt{2} \: x_{\text{rms}}} \sin{(\omega_0 t + \phi)} \\
\color{Blue} + \frac{\delta y(t)}{\sqrt{2} \: x_{\text{rms}}} \cos{(\omega_0 t + \phi)}
\label{Eq:deltaphi}
\end{multline}

\section{Phase Noise Power Spectrum}
%===================================

Taking the Fourier transform of $\delta \phi(t)$, and switching frequency units
\begin{multline*}
\widehat{\delta \phi}(f) = \frac{1}{\sqrt{2} \: x_{\text{rms}}}
\int_{-\infty}^{+\infty} dt \: e^{j \: 2 \pi f t} (- \delta x(t)) \\
\frac{1}{2 j} \left( e^{j \: 2 \pi f_0 t} e^{j \: \phi} - e^{-j \: 2 \pi f_0 t} e^{-j \: \phi} \right) \\
+ \frac{1}{\sqrt{2} \: x_{\text{rms}}}
\int_{-\infty}^{+\infty} dt \: e^{j \: 2 \pi f t} (\delta y(t)) \\
\frac{1}{2} \left( e^{j \: 2 \pi f_0 t} e^{j \: \phi} + e^{-j \: 2 \pi f_0 t} e^{-j \: \phi} \right)
\end{multline*}
Which can be simplified to
\begin{multline}
\widehat{\delta \phi}(f) = \frac{1}{\sqrt{2} \: x_{\text{rms}}}
\left( -\frac{e^{j \: \phi}}{2 j} \: \widehat{\delta x}(f+f_0) + \frac{e^{-j \: \phi}}{2 j} \: \widehat{\delta x}(f-f_0) \right. \\
\left. + \frac{e^{j \: \phi}}{2} \: \widehat{\delta y}(f+f_0) + \frac{e^{-j \: \phi}}{2} \: \widehat{\delta y}(f-f_0) \right) \label{Eq:delta_phi_intermediate}
\end{multline}
We can eliminate $\widehat{\delta y}$ from Eq.~\ref{Eq:delta_phi_intermediate} be recognizing
\begin{subequations}
\label{Eq:delta_y_simp}
\begin{align}
\widehat{\delta y}(f+f_0) & = \widehat{H}(f+f_0) \: \widehat{\delta x}(f+f_0) \nonumber \\
 & = -\frac{1}{j} \: \widehat{\delta x}(f+f_0) \\
\widehat{\delta y}(f-f_0) & = \widehat{H}(f-f_0) \: \widehat{\delta x}(f-f_0) \nonumber \\
 & = \frac{1}{j} \: \widehat{\delta x}(f-f_0)
\end{align}
\end{subequations}
which holds for frequencies $f \leq f_0$, which is the case here.  Substituting Eqs.~\ref{Eq:delta_y_simp} into Eq.~\ref{Eq:delta_phi_intermediate} gives
\begin{multline}
\color{Blue} \widehat{\delta \phi}(f) = - \frac{1}{j} \frac{1}{\sqrt{2} \: x_{\text{rms}}} \times \\
\color{Blue} \left( e^{j \: \phi} \: \widehat{\delta x}(f+f_0) + e^{-j \: \phi} \: \widehat{\delta x}(f-f_0) \right)
\label{Eq:FTdeltaphi}
\end{multline}

Passing to the power spectrum requires a limiting procedure, as follows.  We should consider that $x(t)$ is only sampled for a finite amount of time $T$, which we can indicate with a subscript: $x(t) \rightarrow x_{T}(t)$ where
\begin{equation}
x_{T}(t) = \begin{cases}
0 & \text{for } t > T \\
x(t) & \text{for } -T \leq t < T \\
0 & \text{for } t < -T
\end{cases}
\label{Eq:xT}
\end{equation}
Equation~\ref{Eq:deltaphi} holds with $\delta x \rightarrow \delta x_T$, $\delta x \rightarrow \delta y_T$, and $\delta \phi \rightarrow \delta \phi_T$.  Time correlation functions are defined in terms of $x_T(t)$, not $x(t)$,
\begin{multline}
C_x(\tau) = \lim_{T \rightarrow \infty} \frac{1}{2 T}
\int_{-T}^{+T} \langle x(t) \: x(t + \tau) \rangle \: dt \\
= \lim_{T \rightarrow \infty} \frac{1}{2 T}
\int_{-\infty}^{+\infty} \langle x_{T}(t) \: x_{T}(t + \tau) \rangle \: dt
\end{multline}
where $\langle \cdots \rangle$ indicates a statistical average.  The manipulations leading to Eq.~\ref{Eq:FTdeltaphi} are still valid with the $T$-subscripted variables, with the result that
\begin{multline}
\widehat{\delta \phi_{T}}(f) = - \frac{1}{j} \frac{1}{\sqrt{2} \: x_{\text{rms}}} \times \\
\left( e^{j \: \phi} \: \widehat{\delta x_{T}}(f+f_0) + e^{-j \: \phi} \: \widehat{\delta x_{T}}(f-f_0) \right)
\label{Eq:FTdeltaphiT}
\end{multline}
The next step to computing the power spectrum is to calculate
\begin{multline}
\widehat{\delta \phi_{T}}(f) \: \widehat{\delta \phi_{T}}^{*}\!\!(f) =
 \frac{1}{2 \: x_{\text{rms}}} \times \\
 \left( e^{j \: \phi} \: \widehat{\delta x_{T}}(f+f_0)
  + e^{-j \: \phi} \: \widehat{\delta x_{T}}(f-f_0) \right) \\
 \left( e^{-j \: \phi} \: \widehat{\delta x_{T}}^{*}\!\!(f+f_0)
  + e^{j \: \phi} \: \widehat{\delta x_{T}}^{*}\!\!(f-f_0) \right)
  \label{Eq:PdeltaphiTintermediate}
\end{multline}
We may now pass to the power spectrum by taking the limit
\begin{equation}
P_{\delta x}(f) = \lim_{T \rightarrow \infty} \frac{1}{2 T} \:
 \widehat{\delta x_{T}}(f) \: \widehat{\delta x_{T}}^{*}\!\!(f)
\end{equation}
with the power spectrum $P_{\delta \phi}(f)$ analogously defined.  Carrying out this limiting procedure on both sides of Eq.~\ref{Eq:PdeltaphiTintermediate} yields
\begin{multline}
P_{\delta \phi}(f) = \frac{1}{2 x_{\text{rms}}^2} \left( P_{\delta x}(f+f_0) + P_{\delta x}(f-f_0) \right) \\
 + \frac{1}{2 x_{\text{rms}}^2} \lim_{T \rightarrow \infty} \frac{1}{2 T} \text{Re} \! \left( \widehat{\delta x_{T}}^{*}\!\!(f-f_0) \: \widehat{\delta x_{T}}(f+f_0) \: e^{j \: 2 \phi} \right)
\end{multline}
where $\text{Re} \! \left( \cdots \right)$ indicates taking the real part.  The last term will not survive statistical averaging over the phase $\phi$ since
\begin{equation}
\frac{1}{2 \pi} \int_{0}^{2 \pi} e^{j \: 2 \phi} \: d\phi = 0
\end{equation}
Implicit in this average is the assumption that $\phi$ is randomly distributed, that is, there is no correlation between the phase of the cantilever and the cantilever noise.  After statistical averaging over $\phi$, the power spectrum of cantilever phase noise becomes
\begin{equation}
\boxed{\color{Blue} P_{\delta \phi}(f) = \frac{1}{2 x_{\text{rms}}^2} \left( P_{\delta x}(f+f_0) + P_{\delta x}(f-f_0) \right)}
\label{Eq:Pdeltaphi}
\end{equation}

\section{Frequency Shift Power Spectrum}
%=======================================

Let us define the instantaneous frequency shift as
\begin{equation}
\delta f(t)= \frac{1}{2 \pi} \frac{d}{d t} \: \delta \phi(t) = \frac{1}{2 \pi} \delta \dot{\phi}
\end{equation}
and the compute the power spectrum of the instantaneous frequency shift.  Let us define $\delta f_{T}(t)$  as in Eq.~\ref{Eq:xT}.  The time-correlation function of the frequency shift is then
\begin{equation}
C_{\delta f}(\tau) = \lim_{T \rightarrow \infty} \: \frac{1}{2 T}
\int_{-\infty}^{+\infty} \langle \delta f_{T}(t) \: \delta f_{T}(t+\tau) \rangle \: dt
\end{equation}
with $C_{\delta \phi}$ defined likewise.  Substituting, and dropping $\langle \cdots \rangle$ for notational convenience,
\begin{equation}
C_{\delta f}(\tau) = \frac{1}{4 \pi^2} \lim_{T \rightarrow \infty} \: \frac{1}{2 T}
\int_{-\infty}^{+\infty} \langle \delta \dot{\phi}_{T}(t) \: \delta \dot{\phi}_{T}(t+\tau) \rangle \: dt
\label{Eq:Cdeltaf}
\end{equation}
The time derivative $\delta \dot{\phi}$ may be computing using its Fourier transform.  With
\begin{equation}
\delta \phi_T(t) = \int_{-\infty}^{+\infty} \widehat{\delta \phi_T}(f) \: e^{-j \: 2 \pi f \: t} \: df
\end{equation}
we can compute the time derivative of the instantaneous phase shift as
\begin{equation}
\delta \dot{\phi}_T(t) = \int_{-\infty}^{+\infty} \widehat{\delta \phi_T}(f) \: (-j \: 2 \pi f) \: e^{-j \: 2 \pi f \: t} \: df
\label{Eq:deltadotphiT}
\end{equation}
If we substitute Eq.~\ref{Eq:deltadotphiT} into Eq.~\ref{Eq:Cdeltaf} and use
\begin{equation}
\int_{-\infty}^{+\infty} e^{-j \: 2 \pi (f^{\prime}+f^{\prime\prime}) t } dt = \delta(f^{\prime}+f^{\prime\prime}),
\end{equation}
where $\delta(t)$ is the Kroenecker delta function, then
\begin{multline}
C_{\delta f}(\tau) = \int_{-\infty}^{+\infty}
f^2 \: e^{j \: 2 \pi f \tau} \times \\
\left\{ \lim_{T \rightarrow \infty} \: \frac{1}{2 T} \: \widehat{\delta \phi_T}(f) \: \widehat{\delta \phi_T}(-f) \right\}
 \: df
\end{multline}
where we have passed the limit into the integral.  Because $\delta \phi_T(t)$ is a real function, $\widehat{\delta \phi_T}(-f) = \widehat{\delta \phi_T}^{*}\!\!(f)$.  The term in braces is thus $P_{\delta \phi}(f)$, the power spectrum of phase fluctuations.   We find
\begin{equation}
C_{\delta f}(\tau) = \int_{-\infty}^{+\infty} f^2 \: P_{\delta \phi}(f) \: e^{j \: 2 \pi f \tau} \: df
\end{equation}
Comparing this to the usual relation between the correlation function and the power spectrum
\begin{equation}
C_{\delta f}(\tau) = \int_{-\infty}^{+\infty} P_{\delta f}(f) \: e^{-j \: 2 \pi f \tau} \: df,
\end{equation}
we see that
\begin{equation}
\boxed{\color{Blue} P_{\delta f}(f) =  f^2 \: P_{\delta \phi}(-f)}
\label{Eq:PdeltafPdeltaphi}
\end{equation}
Substituting Eq.~\ref{Eq:PdeltafPdeltaphi} into Eq.~\ref{Eq:Pdeltaphi} we conclude
\begin{equation}
\boxed{\color{Blue} P_{\delta f}(f) =
\frac{f^2}{2 x_{\text{rms}}^2} \left( P_{\delta x}(f_0+f) + P_{\delta x}(f_0-f) \right)}
\label{Eq:Pdeltafresult}
\end{equation}
where we have used that $P_{\delta x}(\Omega) = P_{\delta x}(-\Omega)$.

\section{Instrument Noise}
%=========================
Equation~\ref{Eq:Pdeltafresult} is a general relation between the position-fluctuation power spectrum and the frequency-fluctuation power spectrum.   The power spectrum of detector noise is typically flat:
\begin{equation}
P_{\delta x}(f_0+f) = P_{\delta x}(f_0-f) \equiv P_{\delta x}^{\text{det}}
\end{equation}
Thus
\begin{equation}
\boxed{\color{Blue} P_{\delta f}^{\text{det}}(f) = \frac{f^2 \: P_{\delta x}^{\text{det}}}{x_{\text{rms}}^2} }
\label{Eq:PdeltaxDet}
\end{equation}
This relation holds whether the power spectra are defined an one-sided or two-sided, as long as the power spectrum is computed consistently on both sides of equation.  We typically work up data using one-sided power spectra.

% \bibliography{dielectricfriction,nonlinear}

\section{Cantilever Noise}
%=========================
We have previously shown that the (one sided) power spectrum of cantilever position fluctuation is
\begin{equation}
P_{\delta x}^{\text{one}}(f) = \frac{2 k_B T}{\pi k Q f_0} \frac{f_0^4}{(f_0^2 - f^2)^2 + f^2 f_0^2 / Q^2}
\end{equation}
where $T$ is temperature, $k_B$ is Boltzmann's constant, and $f_0$, $k$, and $Q$ are cantilever frequency, spring constant, and mechanical quality factor, respectively.  We can see that, for frequencies $f \gg f_0 / Q$
\begin{equation}
P_{\delta x}^{\text{one}}(f_0 \pm f) \approx  \frac{2 k_B T}{\pi k Q f_0} \times \frac{f_0^2}{4 f^2}
\end{equation}
Substituting this result into Eq.~\ref{Eq:Pdeltafresult} gives
\begin{equation}
P_{\delta x}^{\text{therm}}(f) = \frac{k_B T f_0}{2 \pi \: x_{\text{rms}}^2 k Q}
\end{equation}
Using
\begin{equation}
Q = \pi f_0 \tau_0,
\end{equation}
where $\tau_0$ is the cantilever ringdown time, we can rewrite the one-sided power spectrum of cantilever frequency fluctuations as
\begin{equation}
\boxed{\color{Blue} P_{\delta x}^{\text{therm}}(f) = \frac{k_B T}{2 \pi^2 \: x_{\text{rms}}^2 k \: \tau_0} }
\label{Eq:PdeltaxTherm}
\end{equation}

\section{Discussion}
%===================

Equations.~\ref{Eq:PdeltaxDet} and \ref{Eq:PdeltaxTherm} agree \emph{exactly} with what Loring and Obukhov et al.~have derived.

\end{document}
