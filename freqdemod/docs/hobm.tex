\input{TechReportTemplate.tex}

\renewcommand{\sect}[1]{\section{#1}}



\begin{document}

\chead{\Large Harmonic Oscillator Brownian Motion}
\lhead{\small -HarmOsc-Brownian-2}
\rhead{\small 20010327 Marohn}

\newcommand{\Real}[1]{\ensuremath{\mbox{Re}\{#1\}}}
\newcommand{\Imag}[1]{\ensuremath{\mbox{Im}\{#1\}}}
\newcommand{\xrms}[0]{\ensuremath{x_{\mbox{\tiny rms}}}}

\sect{Introduction}
% =======================

This report discusses the steady-state response of a cantilever to
both a coherent and an incoherent driving force.  The cantilever is
modeled classically as a damped harmonic oscillator.

In section \ref{sect:equation-of-motion} the equation of motion for a
damped harmonic oscillator is summarized.  When a coherent force is
driving the oscillator, the solution of the equation of motion is
greatly simplified by recasting the equation in terms of a complex
variable whose real and imaginary parts track the in phase and out of
phase response.  This is done in section \ref{sect:phasors}.  The
steady-state response of a damped harmonic oscillator is derived in
section \ref{sect:steady-state-response-I}.

Understanding the response of a damped oscillator to an incoherent
driving force is greatly simplified by thinking in terms of correlation
functions, introduced in section \ref{sect:correlation-functions-I}.
This section introduces correlation functions and establishes a link
between the time-domain correlation function and the frequency-domain
power spectrum.  In section \ref{sect:correlation-functions-II}, a
slightly more sophisticated correlation function is introduced which
is more suitable for understanding physical phenomena.

The steady-state-response of a damped harmonic oscillator to an
incoherent is derived in section \ref{sect:steady-state-response-II}.
The solution relates the power spectrum of the response to the power
spectrum of the incoherent force, and is examined in the limit where
the incoherent driving force has a flat frequency content.  In section
\ref{sect:analyzing-data} the fitting of a cantilever power-spectrum
is discussed.

In the case where the incoherent driving force represents the
oscillator in thermal contact with a bath of modes at a definite
temperature, the form of the force can be derived using the
equipartition theorem of statistical mechanics.  This is accomplished
in section \ref{sect:equipartition-theorem}, where it is also shown that if
oscillator temperature is known, then from the area under the cantilever
power spectrum the cantilever spring constant can be determined.

In section \ref{sect:minimum-detectable-force}, it is shown that the
thus derived thermomechanical cantilever fluctuations limit the
minimum applied force that can be detected by measuring cantilever
displacement.  Designing an experiment to achieve the smallest
possible detectable force is discussed.  In section
\ref{sect:cantilever-design} a scaling analysis is done to 
show the dependence of the minimum detectable force on cantilever
parameters such as length, width, and thickness.

\sect{Equation of motion}
% =======================
\label{sect:equation-of-motion}

The equation of motion for a {\bf damped harmonic oscillator} is
\begin{equation}
m \: \ddot{x} + \alpha \: \dot{x} + k \: x = F
\label{eq:HO} 
\end{equation}

The variables are

\[
\begin{array}{lll}
 x & \mbox{oscillator position} & [\meter] \\ 
 m & \mbox{mass} & [\kilo \gram] \\
 \alpha & \mbox{friction parameter} & [\kilo \gram \: {\second}^{-1} = \newton \: \second \: {\meter}^{-1}] \\
 k & \mbox{spring constant} & [\newton \: {\meter}^{-1}] \\
 F & \mbox{applied force} & [\newton]
\end{array}
\]

It is useful to rewrite this equation in a more canonical form.
Divide eq.~\ref{eq:HO} by $m$, and define new variables according to
the following equations.
\begin{equation}
\frac{\alpha}{m} = \frac{\omega_0}{Q}, \: \: \frac{k}{m} = \omega_0^2, \: \mbox{and} \: \frac{F}{m} = \frac{\omega_0^2 F}{k} = A
\end{equation}

The new variables are

\[
\begin{array}{lll}
 \omega_0 &\mbox{resonance frequency} & [\rad \per \sec] \\
 Q & \mbox{quality factor} & [\mbox{unitless}] \\
 A & \mbox{applied force amplitude} & [\meter \: \second^{-2}]
\end{array}
\]

The canonical equation of motion for a classical harmonic oscillator is thus
\begin{equation}
\ddot{x} + \frac{\omega_0}{Q} \: \dot{x} + \omega_0^2 \: x = A = \frac{\omega_0^2 \: F}{k}
\label{eq:HO-canonical}
\end{equation}


\sect{Phasors}
% ============
\label{sect:phasors}

We wish to calculate the response of the oscillator to a resonant
force,
\[
F(t) \propto \cos{\omega t}
\]
Here $\omega$ is the driving frequency, close to but not necessarily
equal to $\omega_0$.  At \emph{steady state}, the cantilever response
must also be periodic, of the general form
\begin{equation}
x(t) = x_c \cos{\omega t} + x_s \sin{\omega t}
\end{equation}
We wish to solve for $x_c$ and $x_s$ as a function of driving frequency.
It is convenient to introduce a complex number $z$ that tracks
cantilever displacement, $x = \Real{z}$.  If we make the ansatz that
$z = z_0 \exp{(\imath \: \omega t)}$ then
\begin{eqnarray}
x(t) & = & \Real{z} = \Real{z_0 \: e^{\imath \: \omega t}} \\
     & = & \underbrace{\Real{z_0}}_{x_c} \cos{\omega t} - \underbrace{\Imag{z_0}}_{x_s} \sin{\omega t}
\end{eqnarray}
If we can recast eq.~\ref{eq:HO-canonical} in terms of the complex
variable $z$ then we can reduce the problem of solving for two real
variables, $x_c$ and $x_s$, to solving for one complex variable,
$z_0$.

Towards this end, we introduce another complex variable $F_c$ which
tracks the applied force.  If the force is a sinusoidal function of
time, then $F_c = F_0 \exp{(\imath \: \omega t)}$ where $F_0 = | F_0 |
\: \exp{(\imath \: \phi)}$ is complex number that describes the
magnitude and phase of the harmonic driving force:
\begin{eqnarray}
F(t) & = & \Real{F_c} = \Real{F_0 \: e^{\imath \: \omega t}} \\
     & = & \Real{F_0} \cos{\omega t} - \Imag{F_0} \sin{\omega t} \\
     & = & | F_0 | \cos{(\omega t + \phi)}
\end{eqnarray}

The equation of motion for $z$ in terms of {\bf phasors} is
\begin{equation}
\ddot{z} + \frac{\omega_0}{Q} \: \dot{z} + \omega_0^2 \: z = \frac{\omega_0^2 \: F_c}{k}
\label{eq:z}
\end{equation}


\sect{Steady-state response I}
% =====================================
\label{sect:steady-state-response-I}

It is convenient to work with frequency in experimental units of
$[\mbox{cyc}/\sec] = [\hertz]$ instead of $[\mbox{rad}/\sec]$.
Therefore we'll define
\begin{equation}
f_0 = \frac{\omega_0}{2 \pi} \: \sim \: [\frac{\mbox{cyc}}{\sec}] = [\hertz]
\end{equation}
and work throughout with frequencies in $\hertz$.

In this section we explore the response of the harmonic oscillator to
a {\bf coherent sinusoidal driving force}.  Substitute $F_c = F_0
\exp{(\imath \: 2 \pi f t)}$ into eq.~\ref{eq:z} 
and assume that the response $z$ is of the form $z_0 \exp{(\imath \: 2 \pi f t)}$:
\begin{equation}
(-f^2 + \imath f \: \frac{f_0}{Q} + f_0^2 ) \: z_0 \: e^{\imath \: 2 \pi f t} = \frac{f_0^2}{k} F_0 \: e^{\imath \: 2 \pi f t}
\end{equation}
where we have canceled a factor of $4 \pi^2$ from every term.  We infer that
\begin{equation}
z_0 = \frac{F_0}{k} \: \frac{f_0^2}{f_0^2 - f^2 + \imath \: f \: f_0 / Q}
\end{equation}
so that at steady state
\begin{eqnarray}
z(f) & = & z_0 \: e^{\imath \: 2 \pi f t} \\
     & = & \frac{F_0 \: e^{\imath \: 2 \pi f t}}{k} \: \frac{f_0^2}{f_0^2 - f^2 + \imath \: f \: f_0 / Q}
\end{eqnarray}
It is useful to write $z$ as follows:
\begin{eqnarray}
\lefteqn{z(f) = \frac{| F_0 |}{k} \left( \frac{f_0^2 (f_0^2 - f^2)}{(f_0^2 - f^2)^2 + f^2 \: f_0^2 / Q^2} \right. \nonumber}\\
&  & \left. - \imath \: \frac{f \: f_0^3 / Q}{(f_0^2 - f^2)^2 + f^2 \: f_0^2 / Q^2} \right) \: e^{\imath \: ( 2 \pi f t + \phi)}
\end{eqnarray}
Using $x = \Real{z}$ we can infer that $x(t)$ is of the form
\begin{equation}
x(t) = x_c \: \cos{(2 \pi f t + \phi)} + x_s \: \sin{(2 \pi f t + \phi)}
\end{equation}
where
\begin{equation}
x_c(f) = \frac{| F_0 |}{k} \frac{f_0^2 ( f_0^2 - f^2)}{(f_0^2 - f^2)^2 + f^2 \: f_0^2 / Q^2} 
\end{equation}
\begin{equation}
x_s(f) = \frac{| F_0 |}{k} \frac{f \: f_0^3 / Q}{(f_0^2 - f^2)^2 + f^2 \: f_0^2 / Q^2}
\end{equation}
The signal $x_c$ is the part of the response detected with a lock-in
as \emph{in phase} with the driving force.  The signal $x_s$ is the
\emph{out of phase} part of the response.

When the applied force drives the oscillator right on resonance,
$\omega = \omega_0$, and we compute that
\begin{eqnarray}
x_s(\omega_0) & = & 0 \\
x_s(\omega_0) & = & \frac{Q \: | F_0 |}{k}
\end{eqnarray}
This is to be compared to the steady-state response to a
non-oscillating (DC) force
\begin{eqnarray}
x_c(0) & = & \frac{| F_0 |}{k} \\
x_s(0) & = & 0
\end{eqnarray}
We conclude that the response to a resonant force is $Q$ times larger
than the response to a static DC force.  The response at resonance is
also ninety degrees out of phase with the applied oscillating force.
We can see this immediately from
\begin{equation}
z_0(\omega_0) = - \imath \: \frac{Q \: F_0}{k}
\end{equation}



\sect{Correlation functions I}
% ======================================
\label{sect:correlation-functions-I}

The section explores a connection between a function's associated
correlation function and power spectrum.  The correlation function of
$x(t)$ is defined as
\begin{equation}
C_x(\tau) = \int_{-\infty}^{\infty} dt \: x(t) \: x(t+\tau) \: \sim \: [\frac{\meter^2}{\hertz}]
\end{equation}

The Fourier and inverse Fourier transforms are taken
as in the following text.

\indent \\
\indent \underline{Numerical Recipes} \\
\indent W. H.  Press, B. P. Flannery, S. A. Teukolsky, \\
\indent and W. T. Vetterling \\
\indent Cambridge University Press, NY (1986) \\

Following Press \emph{et al.}, the Fourier and inverse Fourier
transforms of $x(t)$ are:

\begin{equation}
\hat{x}(f) = \int_{-\infty}^{\infty} dt \: x(t) \: e^{\imath \: 2 \pi f t}
\end{equation}
\begin{equation}
x(t) = \int_{-\infty}^{\infty} df \: \hat{x}(f) \: e^{-\imath \: 2 \pi f t}
\end{equation}
Substitute for $x(t)$ and $x(t+\tau)$ the appropriate Fourier
transform relation
\begin{eqnarray*}
\lefteqn{C_x(\tau) = \int df  \int df^{\prime}  \: \hat{x}(f^{\prime}) \: \hat{x}(f) \: e^{-\imath \: 2 \pi f \tau}}\\
& & \underbrace{\int dt \: e^{-\imath \: 2 \pi f t}  e^{-\imath \: 2 \pi f^{\prime} t}}_{\delta(f+f^{\prime}) \Longrightarrow f^{\prime} = -f} 
\end{eqnarray*}
The integral over time involving exponentials reduces to a delta
function.  Only frequencies $f^{\prime} = -f$ contribute to the final
double integral, so that
\[
C_x(\tau) = \int_{-\infty}^{\infty} df \: \: \hat{x}(-f) \: \hat{x}(f) \: e^{-\imath \: 2 \pi f \tau}
\]
If $x(t)$ is a real function of time, then it can be shown that
$\hat{x}(-f) = \hat{x}^{*}(f)$ where the star indicates the complex
conjugate.  We have finally
\begin{eqnarray}
C_x(\tau) & = & \int_{-\infty}^{\infty} df \: \hat{x}^{*}(f) \: \hat{x}(f) \: e^{-\imath \: 2 \pi f \tau} \\ 
          & = & \int_{-\infty}^{\infty} df \: | \hat{x}(f) |^2 \: e^{-\imath \: 2 \pi f \tau} 
\end{eqnarray}
This is an important result: 
\begin{quote}
  \emph{The correlation function and the power spectrum are Fourier
  transform pairs.}
\end{quote}

If we define the one-sided power spectral density as
\begin{equation}
\hat{P}_x(f) = | \hat{x}(f) |^2 + | \hat{x}(-f) |^2  \: \sim \: [\frac{\meter^2}{\hertz^2}]
\end{equation}
then
\begin{equation}
C_x(\tau) = \int_{0}^{\infty} df \: \hat{P}_x(f) \: e^{-\imath \: 2 \pi f \tau}
\end{equation}


\sect{Correlation functions II}
% =======================================
\label{sect:correlation-functions-II}

The correlation function considered above is not suitable for
considering physical phenomena.  The physically-relevant correlation
function is treated in

\indent \\
\indent Chapter 1 \\
\indent \underline{Photon-Atom Interactions} \\
\indent Mitchel Weissbluth \\
\indent Academic Press, NY (1989) \\

Following Weissbluth, we define the correlation function as follows.

\begin{equation}
G(\tau) \equiv \langle x(t) x(t+\tau) \rangle
\end{equation}
\begin{equation}
G(\tau) \equiv \lim_{T \rightarrow \infty} \: \frac{1}{2 T} \int_{-T}^{+T} x(t) x(t+\tau) \: dt \: \sim \: [\meter^2]
\label{eq:CF}
\end{equation}

The units of this correlation function are $[\meter^2]$, if the units
of x are $[\meter]$.  This is quite different from the
mathematically-defined correlation function $C(\tau)$ above, whose
units are $[\meter^2 \per \hertz]$.

The correlation function at $\tau=0$, zero delay, has special
significance:
\begin{equation}
G(0) = \lim_{T \rightarrow \infty} \: \frac{1}{2 T} \int_{-T}^{+T} x^2(t) \: dt = \xrms^2
\end{equation}
Thus $G(0)$ is the square of the root-mean-square value of $x(t)$ and
$\xrms = \sqrt{G(0)}$.

We will now reproduce Weissbluth's treatment relating the
(physically-relevant) correlation function $G(\tau)$ to an analogous
power spectrum.  So following Weissbluth, define the function
$x_{T}(t)$ which is equal to $x(t)$ on the time interval $(-T,+T)$ and
is zero at all other times:
\begin{equation}
x_{T}(t) = \left\{ \begin{array}{cc} x(t) & -T \leq t \leq +T \\ 0 & \mbox{otherwise} \end{array} \right.
\end{equation}
Define too a correlation function for $x_T$ as follows.
\begin{eqnarray}
G_{T}(\tau) & = & \frac{1}{2 T} \int_{-T}^{+T} x_T(t) x_T(t+\tau) \: dt \\
            & = & \frac{1}{2 T} \int_{-\infty}^{+\infty} x_T(t) x_T(t+\tau) \: dt
\end{eqnarray}
Since we've confined $x_T$ to the time interval $(-T,+T)$ we can
extend the limits in integration out to infinity.  Now take the
Fourier transform of $G_{T}(\tau)$:
\begin{eqnarray*}
\lefteqn{\int_{-\infty}^{+\infty} G_{T}(\tau) \: e^{\imath \: 2 \pi f \tau} \: d\tau =} \\
& = & \frac{1}{2 T} \int_{-\infty}^{+\infty} d\tau \: e^{\imath \: 2 \pi f \tau} \int_{-\infty}^{+\infty} dt \: x_{T}(t) \: x_{T}(t+\tau) \\
& = & \frac{1}{2 T} \int_{-\infty}^{+\infty} dt \: x_{T}(t) \: e^{-\imath \: 2 \pi f t} \\
&   & \hspace{0.5in}\int_{-\infty}^{+\infty} d\tau \:  x_{T}(t+\tau) \: e^{\imath \: 2 \pi f (t+\tau)}
\end{eqnarray*}
where we have inserted 1 in the form of $\exp{(-\imath \: 2 \pi f t)}
\exp{(+\imath \: 2 \pi f t)}$.  In the second integral, change the
variable of integration to $t^{\prime} = t+\tau$.  This lets us write
\begin{eqnarray*}
\lefteqn{\int_{-\infty}^{+\infty} G_{T}(\tau) \: e^{\imath \: 2 \pi f \tau} \: d\tau =} \\
& = & \frac{1}{2 T} \underbrace{\int_{-\infty}^{+\infty} dt \: x_{T}(t) \: e^{-\imath \: 2 \pi f t}}_{{\hat{x}}_T(-f) = {\hat{x}}^{*}_{T}(f)} \\
&   & \hspace{0.5in} \underbrace{\int_{-\infty}^{+\infty} dt^{\prime} \:  x_{T}(t^{\prime}) \: e^{\imath \: 2 \pi f t^{\prime}}}_{{\hat{x}}_T(f)}
\end{eqnarray*}
Since $x(t)$ is a real function, it follows that ${\hat{x}}_{T}(-f) = {\hat{x}}^{*}_{T}(f)$.
This allows us to write
\begin{equation}
\int_{-\infty}^{+\infty} G_{T}(\tau) \: e^{\imath \: 2 \pi f \tau} \: d\tau = \frac{1}{2 T} \: | \hat{x}(f) |^{2}
\label{eq:limitG}
\end{equation}
We recover the ``real'' correlation function by a limiting procedure.
\begin{equation}
G(\tau) = \lim_{T \rightarrow \infty} \: G_{T}(\tau)
\end{equation}
Take the limit on each side of eq.~\ref{eq:limitG} as $T \rightarrow
\infty$.  On the LHS $G_T$ becomes $G$; the terms on the RHS motivate
us to define
\begin{equation}
J(f) \equiv \lim_{T \rightarrow \infty} \: \frac{1}{2 T} \: | \hat{x}(f) |^{2} \: \sim \: [\frac{\meter^2}{\hertz}]
\label{eq:PS}
\end{equation}
as the \emph{physically relevant spectral density}.  It still holds that
\begin{equation}
J(f) = \int_{-\infty}^{+\infty} G(\tau) \: e^{\imath \: 2 \pi f \tau} \: d\tau
\end{equation}
and
\begin{eqnarray}
G(\tau) & = & \int_{-\infty}^{+\infty} J(f) \: e^{-\imath \: 2 \pi f \tau} \: df \\
        & = & \int_{0}^{+\infty} P(f) \: e^{-\imath \: 2 \pi f \tau} \: df.
\label{eq:FTOSPS}
\end{eqnarray}
We have defined the one-sided power spectral density as
\begin{eqnarray}
P(f) & = & J(f) + J(-f) \\
     & = & \lim_{T \rightarrow \infty} \frac{1}{2 T} \: ( | \hat{x}(f) |^{2} + | \hat{x}(-f) |^{2})
\label{eq:OSPS}
\end{eqnarray}

With these definitions of correlation function (eq.~\ref{eq:CF}) and
spectral density (eq.~\ref{eq:PS}), we still have that
\begin{quote}
  \emph{The correlation function $G(\tau)$ and the power spectrum $J(f)$ of $x(t)$ are Fourier
  transform pairs.}
\end{quote}

Finally, eq.~\ref{eq:FTOSPS} can be used to calculate the
root-mean-square of $x(t)$ given a measured one-sided power spectral
density:
\begin{equation}
\xrms^2 = \langle x^2(t) \rangle = G(0) = \int_{0}^{+\infty} P(f) \: df.
\label{eq:xrmsP}
\end{equation}
We conclude that
\begin{quote}
   \emph{The area under the one-sided spectrum is the mean-square
   displacement}.
\end{quote}
We note that this connection is not valid for the
mathematically-defined power-spectrum of the last section.


\sect{Steady-state response II}
% =============================
\label{sect:steady-state-response-II}

In this section we explore the response of the harmonic oscillator to
an {\bf incoherent} driving force.  If the force is random, it will
have zero average:
\begin{equation}
\langle F(t) \rangle = \lim_{T \rightarrow \infty} \: \frac{1}{2 T} \int_{-T}^{+T} F(t) \: dt \longrightarrow 0
\end{equation}
It will not, in general, have a vanishing correlation function -- we
will discuss the force and response using correlation functions.
Integrating eq.~\ref{eq:z} provides another route to understanding the
response $z(t)$ to a randomly fluctuating force $F(t)$ driving the
system -- we will not follow such a Langevin treatment.

Define correlation functions for $z$ and $F$ as above.
\begin{equation}
G_z(\tau) \equiv \lim_{T \rightarrow \infty} \: \frac{1}{2 T} \int_{-T}^{+T} z(t) z(t+\tau) \: dt \: \sim \: [\meter^2]
\end{equation}
\begin{equation}
G_F(\tau) \equiv \lim_{T \rightarrow \infty} \: \frac{1}{2 T} \int_{-T}^{+T} F(t) F(t+\tau) \: dt \: \sim \: [\newton^2]
\end{equation}
With each of these correlation functions is associated a power spectrum:
\begin{eqnarray*}
G_z(\tau) & \Leftarrow \mbox{FT} \Rightarrow & J_z(f) \: \mbox{or} \: P_z(f) \\
G_F(\tau) & \Leftarrow \mbox{FT} \Rightarrow & J_F(f) \: \mbox{or} \: P_z(f)
\end{eqnarray*} 
Because $z$ and $F$ are connected by an equation of motion, we can
write $J_z$ in terms of $J_F$, as we will now show.

Follow the motion by Fourier analysis:
\begin{eqnarray}
F(t) & = & \int_{-\infty}^{\infty} df \: \hat{F}(f) \: e^{-\imath \: 2 \pi f t} \label{eq:FTF}\\
z(t) & = & \int_{-\infty}^{\infty} df \: \hat{z}(f) \: e^{-\imath \: 2 \pi f t} \label{eq:FTz}
\end{eqnarray}
Substitute eq.~\ref{eq:FTF} and eq.~\ref{eq:FTz} into the equation of
motion connecting $F$ and $z$, eq.~\ref{eq:z}.
\begin{eqnarray*}
\lefteqn{\int_{-\infty}^{+\infty} (-f^2 - \imath f \: \frac{f_0}{Q} + f_0^2 ) \: \hat{z}(f) \: e^{-\imath \: 2 \pi f t} \: df} \\
& & \hspace{0.5in} = \int_{-\infty}^{+\infty} \frac{f_0^2}{k} \hat{F}(f) \: e^{-\imath \: 2 \pi f t} \: df
\end{eqnarray*}
For both sides to be equal, we must have that at each frequency
\begin{equation}
\hat{z}(f) = \frac{\hat{F}(f)}{k} \frac{f_0^2}{f_0^2 - f^2 - \imath f \: f_0 / Q}
\end{equation}
Taking the magnitude of each side, we infer that the power spectra are
related by
\begin{equation}
| \hat{z}(f) |^2 = \frac{| \hat{F}(f) |^2}{k^2} \frac{f_0^4}{(f_0^2 - f^2)^2 + f^2 f_0^2 / Q^2}
\end{equation}

This equation relates ``mathematical'' correlation functions.  It is a
straightforward matter to introduce the time-averaging and limiting
procedure employed above to obtain this result in terms of
``physically-relevant'' correlation functions:  
\begin{equation}
P_z(f) = \lim_{T \rightarrow \infty} \frac{1}{2 T} \: ( | \hat{z}(f) |^{2} + | \hat{z}(-f) |^{2}) \: \sim \: [\frac{\meter^2}{\hertz}]
\end{equation}
\begin{equation}
P_F(f) = \lim_{T \rightarrow \infty} \frac{1}{2 T} \: ( | \hat{F}(f) |^{2} + | \hat{F}(-f) |^{2}) \: \sim \: [\frac{\newton^2}{\hertz}]
\label{eq:PF}
\end{equation}
The result, which we write in terms of \emph{one-sided power spectral
densities} is:
\begin{equation}
P_z(f) = \frac{P_F(f)}{k^2} \frac{f_0^4}{(f_0^2 - f^2)^2 + f^2 f_0^2 / Q^2}
\label{eq:PzPF}
\end{equation}

Given an $F(t)$, form a one-sided power spectrum $P_F(f)$ by Fourier
transforming the time-domain spectrum of $F$ and averaging
(eq.~\ref{eq:PF}).  We can then predict the resulting one-sided power
spectrum $P_z(f)$ of the response $z(t)$ using eq.~\ref{eq:PzPF}.
Finally, if we wish, we could determine what would be the
time-correlation function $G_z(\tau)$ of $z(t)$.

We can proceed no further in discussing the response of the harmonic
oscillator to an incoherent driving force unless we specify a form for
either $F(t)$, $G_F(\tau)$, $J_F(f)$, or the power spectrum $P_F(f)$.
The simplest approximation is to assume that the force fluctuation
driving the oscillator is well-described as being \emph{white noise},
e.g., a randomly-fluctuating with a power spectrum that is flat up to
some very high frequency cutoff:
\begin{equation}
P_F(f) = \left\{ \begin{array}{cc} P_F(0) & 0 \leq f \leq f_m \\ 0 & f_m \leq f \end{array} \right.
\label{eq:whitenoise}
\end{equation}

The cutoff frequency's numerical value is determined by the physical
process giving rise to the force fluctuation.  Atomic force microscope
cantilevers experience force fluctuations due to random collisions
with gas molecules and fluctuating cantilever phonon populations, for
example.  Both of these processes have characteristic timescales on
the order of nanoseconds, which implies (by Fourier transform of the
associated correlation function) that $f_m \sim 1 / \nano\sec = \giga
\hertz$.  

Atomic force cantilever resonance frequencies are in the range of $f_0
\sim 1 - 500 \: \kilo \hertz$, so that $f_0 << f_m$, and thus when considering a cantilever's 
response to the above-mentioned force fluctuations the approximation
of eq.~\ref{eq:whitenoise} is a good one.  An example of a case where
the white-noise approximation would not be valid is the cantilever
being driven by acoustic room vibrations.  The power spectrum of doors
closing, mechanical vibrations from transformers, and people walking
by the cantilever is generally not flat near the cantilever resonance
frequency.

If the cantilever is being driven by white noise, then
\begin{equation}
P_z(f) = \underbrace{\frac{P_F(0)}{k^2}}_{\mbox{\small freq. independent}} 
\underbrace{\frac{f_0^4}{(f_0^2 - f^2)^2 + f^2 f_0^2 / Q^2}}_{\mbox{\small freq. \emph{dependent}}}
\label{eq:PzPFconst}
\end{equation}


\sect{Analyzing Data}
% ==========================
\label{sect:analyzing-data}

As a practical matter, the the position fluctuation is fit to:
\begin{equation}
P_z(f) = P_z(0) \underbrace{\frac{f_0^4}{(f_0^2 - f^2)^2 + f^2 f_0^2 / Q^2}}_{\mbox{\small unitless}} + P_x^{\mbox{\tiny noise}}
\label{eq:Pzfit}
\end{equation}
The first term is the power spectrum of the cantilever, the form of
which we derived above, and the second term represents detector noise.
Here
\begin{equation}
P_z(0) = \frac{P_F(0)}{k^2} \: \sim \: [\frac{\meter^2}{\hertz}]
\label{eq:Pz0}
\end{equation}
is the apparent position fluctuation at zero frequency.  If the
cantilever and instrument-noise related fluctuations are uncorrelated
-- a good assumption -- then the power spectrums just add.

Over a narrow bandwidth centered at the cantilever frequency, the
instrument noise power spectrum $P_x^{\mbox{\tiny noise}}$ can often
be approximated as constant.  If working with a low-Q cantilever near
zero-frequency, ``$1/f$''instrument noise begins to contribute.  In
this case, the ``$1/f$'' component can often be well-approximated by adding
a linear term:
\begin{equation}
P_x^{\mbox{\tiny noise}} \approx P^{(0)} + P^{(0)} (f - f_0)
\end{equation}
Here $P^{(0)} \: \sim \: [\meter^2/\hertz]$ is the
frequency-independent term and $P^{(1)} \: \sim \:
[\meter^2/\hertz^2]$ approximates frequency-dependent noise sources,
including ``$1/f$'' circuit noise.

By fitting the observed $P_z(f)$ to eq.~\ref{eq:Pzfit}, the cantilever
resonance frequency $f_0$ and quality factor $Q$ may be determined.  If
$k$ is known, the force fluctuation power spectral density can be
inferred using eq.~\ref{eq:Pz0}.  If the force fluctuations are
described by a bath of modes at a well defined \emph{temperature},
then statistical mechanics constrains what $P_F(0)$ \emph{must} be, as will
now be discussed.


\sect{Equipartition theorem}
% ==========================
\label{sect:equipartition-theorem}

As may be derived using statistical mechanics, a harmonic oscillator
in equilibrium with a bath of temperature $T$ has a energy expectation
value for each mode equal to $k_B T/2$.  Thus
\begin{equation}
\frac{1}{2} \: k \langle x^2 \rangle = \frac{1}{2} \: k_B T
\label{eq:equip}
\end{equation}
where $k_B = 1.38 \: \times \: {10}^{-23} \: \joule \: {\kelvin}^{-1}$
is Boltzmann's constant and $T \: [\kelvin]$ is the absolute
temperature.  Here $\langle x^2 \rangle$ is mean-square displacement $x_{\mbox{\tiny rms}}^2$.  If the
oscillator is in thermal equilibrium with a bath described by a
temperature $T$, then if $x_{\mbox{\tiny rms}}^2$ can be measured, the
oscillator spring constant can be inferred from
\begin{equation}
k = \frac{k_B T}{x_{\mbox{\tiny rms}}^2} \: \sim \: [\frac{\newton}{\meter}]
\label{eq:k}
\end{equation}

The mean-square displacement can be measured directly from time-domain
observations.  An alternative and more accurate way to determine
$\xrms$ is to employ eq.~\ref{eq:xrmsP} and calculate $\xrms$ as the
area under the position-fluctuation power spectrum.  In practice both circuit
noise and cantilever fluctuations contribute to the power spectrum,
and therefore, by eq.~\ref{eq:xrmsP}, to the observed time-domain
$\xrms$. Having fit data to eq.~\ref{eq:Pzfit}, the integral of the
cantilever's contribution to the power spectrum may be calculated
analytically in from the fit parameters as follows (see the
appendix):
\begin{eqnarray}
\xrms^2 & = & P_z(0) f\: _0^4 \: (\int_{0}^{\infty} df \frac{1}{(f^2 - f_0^2)^2 + f^2 f_0^2 / Q^2}) \nonumber \\
        & = & \frac{\pi}{2} \: P_z (0) \: Q \: f_0 \label{eq:xrmscalc}
\end{eqnarray}

Having thus employed correlation-function results to accurately
$\xrms$, the spring constant my be inferred.  Substituting
eq.~\ref{eq:xrmscalc} into eq.~\ref{eq:k} gives the desired relation
\begin{equation}
k = \frac{2 \: k_B T}{\pi P_z(0) \: Q \: f_0} \: \sim \: [\frac{\newton}{\meter}]
\label{eq:k2}
\end{equation}



\sect{Minimum detectable force}
% =============================
\label{sect:minimum-detectable-force}

We can turn eq.~\ref{eq:k2} around to read
\begin{equation}
P_z(0) = \frac{2 \: k_B T}{\pi k Q f_0} \: \sim \: [\frac{\meter^2}{\hertz}]
\end{equation}
\begin{quote}
   \emph{This is what statistical mechanics says $P_z(0)$ must be for
   a harmonic oscillator in thermal equilibrium with a bath at
   temperature $T$}
\end{quote}
if the harmonic oscillator is to satisfy the equipartition theorem
(eq.~\ref{eq:equip}).  The power spectral density at all frequencies
for a harmonic oscillator at thermal equilibrium is obtained by
substituting this $P_z(0)$ into eq.~\ref{eq:Pzfit}:
\begin{equation}
P_z(f) =  (\frac{2 \: k_B T}{\pi k Q f_0})(\frac{f_0^4}{(f_0^2 - f^2)^2 + f^2 f_0^2 / Q^2})
\end{equation}
The first term in parenthesis has units of $[\meter^2/\hertz]$ and
serves to fix the area under the power spectrum.  The second term is
unitless and traces out the response versus frequency of the
oscillator to thermal-bath fluctuations.

We can infer the thermal force-fluctuation spectral density using
$P_F(0) = k^2 P_z(0)$.  The answer is
\begin{equation}
P_F(0) = \frac{2 \: k \: k_B T}{\pi Q f_0} \: \sim \: [\frac{\newton^2}{\hertz}]
\label{eq:PF0}
\end{equation}
Thermal cantilever position fluctuations can be treated as if due to a
\emph{force} fluctuation of this spectral density.

At resonance
\begin{equation}
P_z(f_0) = (\frac{2 \: k_B T}{\pi k Q f_0})(Q^2) = \frac{2 \: Q \: k_B T}{\pi k f_0} \: \sim \: [\frac{\meter^2}{\hertz}] 
\end{equation}
We are interested in the position-noise power in a narrow bandwidth $\Delta \!
f$ centered at the oscillator resonance frequency $f_0$, such as would
be measured with a lock-in amplifier.  The noise power is:
\begin{eqnarray*}
x_{\mbox{\tiny min}}^2(f_0) & = & \int_{f_0 - \Delta \! f / 2}^{f_0 + \Delta \! f / 2} P_z(f) \: df \approx P_z(f_0) \int_{f_0 - \Delta \! f/2}^{f_0 + \Delta \! f/2} df\\  
& = & \frac{2 \: Q \: k_B T}{\pi k f_0} \times \Delta \! f \: \sim \: [\meter^2]
\end{eqnarray*}
The root-mean-square detectable position at resonance is the square
root of this quantity:
\begin{equation}
x_{\mbox{\tiny min}}(f_0) = \sqrt{ \frac{2 \: Q \: \Delta \! f \: k_B T}{\pi k f_0} } \: \sim \: [\meter]
\end{equation}
It is interesting to calculate the position-noise power in a narrow
bandwidth centered at \emph{zero} frequency.  Calculate:
\begin{equation}
x_{\mbox{\tiny min}}^2(0) \approx P_z(0) \: \Delta \! f 
= \frac{2 \: k_B T}{\pi k Q f_0} \times \Delta \! f \: \sim \: [\meter^2]
\end{equation}
As we expect, there is less power in fluctuations far away from
resonance.  For completeness, the zero-frequency root-mean-square
detectable position is:
\begin{equation}
x_{\mbox{\tiny min}}(0) = \sqrt{ \frac{2 \: \Delta \! f \: k_B T}{\pi k Q f_0} } \: \sim \: [\meter]
\end{equation}

The minimum detectable force is inferred from the force-noise power
in a narrow band of frequency near resonance:
\begin{eqnarray*}
F_{\mbox{\tiny min}}^2 & = & \int_{f_0 - \Delta \! f / 2}^{f_0 + \Delta \! f / 2} P_F(f) \: df =  P_F(0) \int_{f_0 - \Delta \! f/2}^{f_0 + \Delta \! f/2} df\\  
& = & \frac{2 \: k \: k_B T}{\pi Q f_0} \times \Delta \! f \: \sim \: [\newton^2]
\end{eqnarray*}
where we have taken $P_F(f) = P_F(0)$ from eq.~\ref{eq:PF0}.  The
root-mean-square detectable force is thus:
\begin{equation}
F_{\mbox{\tiny min}} = \sqrt{ \frac{2 \: k \: \Delta \! f \: k_B T}{\pi Q f_0} } \: \sim \: [\newton]
\label{eq:Fmin}
\end{equation}
Note that the $x_{\mbox{\tiny min}}$ calculated above is only valid
near resonance, whereas eq.~\ref{eq:Fmin} for $F_{\mbox{\tiny min}}$ is
valid at \emph{all frequencies}.

It is convenient to write $x_{\mbox{\tiny min}}$ in terms of a
position-fluctuation spectral density at resonance $S_x \sim [\meter
\hertz^{-1/2}]$ times the square root of the detection bandwidth, as follows.  
Similarly $F_{\mbox{\tiny min}}$ can be recast in terms of a
force-fluctuation spectral density $S_F \sim [\newton \hertz^{-1/2}]$.
\begin{eqnarray}
x_{\mbox{\tiny min}} & = & S_x \: \sqrt{\Delta \! f} \\
F_{\mbox{\tiny min}} & = & S_F \: \sqrt{\Delta \! f}
\end{eqnarray}
Here the position- and force-fluctuation spectral density near resonance are:
\begin{eqnarray}
S_x & = & \sqrt{ \frac{2 \: Q \: k_B T}{\pi k f_0} } \: \sim \: [\frac{\meter}{\sqrt{\hertz}}] \\
S_F & = & \sqrt{ \frac{2 \: k \: k_B T}{\pi Q f_0} } \: \sim \: [\frac{\newton}{\sqrt{\hertz}}] \label{eq:SF}
\end{eqnarray}

The quantity $S_F$ is an especially useful figure of merit for force
detection near resonance; it allows one to compare cantilevers without
specifying a detection bandwidth.  Equation~\ref{eq:SF} makes clear
what is required for best force sensitivity:
\begin{itemize}
\item lowest possible spring constant $k$
\item lowest possible temperature $T$
\item highest possible quality factor $Q$
\item highest possible resonance frequency $f_0$
\end{itemize}

Rewrite $S_F$ by substituting $k = 4 \pi^2 f_0^2 m$ and writing $Q =
\tau f_0$ where $\tau$ here is the cantilever damping time.  This
recasts $S_F$ as
\begin{equation}
S_F = \sqrt{ 8 \pi \: k_B T \: \frac{m}{\tau} \: \Delta \! f}
\end{equation}
Another way to achieve the best possible force sensitivity is to:
\begin{itemize}
\item work at the lowest possible temperature $T$
\item minimize cantilever motional mass $m$
\item maximize cantilever damping times $\tau$
\end{itemize}


\sect{Cantilever design}
% ==========================
\label{sect:cantilever-design}

The resonance frequency and spring constant for a beam cantilever of
length $l$, width $w$, and thickness $t$ are:
\begin{equation}
f_0 = \frac{3.516}{2 \pi} \frac{t}{l^2} \left( \frac{E}{12 \rho} \right)^{1/2}
\end{equation}
\begin{equation}
k = 1.030 \frac{l}{4} \frac{E w t^3}{l^3}
\end{equation}
where $E$ is Young's modulus and $\rho$ is density ($E = 1.9 \times
10^{11} \: \newton \: \meter^{-2}$ and $\rho = 2.3 \times 10^{3} \:
\kilo\gram \: \meter^{-3}$ for silicon).  In terms of cantilever properties,
\begin{equation}
S_F = 1.588 \left( \frac{k_B T}{Q} \right)^{1/2} (\rho E)^{1/4} \left( \frac{w}{l} \right)^{1/2} t 
\end{equation}
The critical cantilever parameter to optimize to achieve the best
possible force sensitivity is thus cantilever thickness $t$.  The next
best cantilever property to optimize is the width to length ratio,
$w/l$.  Finally, cantilever material density and Young's modulus, because they
appear in $S_F$ to the 1/4 power, are the least important parameters
to optimize.



\sect{Appendix: an integral}
% ==========================
\label{sect:appendix-an-integral}

We wish to compute the following integral
\begin{equation}
P = P_z(0) \: f_0^4 \int_{0}^{\infty} df \frac{1}{(f^2 - f_0^2)^2 + f^2 f_0^2 / Q^2}
\end{equation}
This integral can be rearranged to resemble an integral found in
standard tables or that Mathematica can solve.  Let
\begin{eqnarray}
f & = & f_0 F \\
df & = & f_0 dF 
\end{eqnarray}
where $F$ is a unitless frequency parameter.  The integral rewritten
in terms of $F$ is
\begin{equation}
P = P_z(0) \: f_0^4 \int_{0}^{\infty} \frac{f_0 \: dF}{(f_0^2 F^2 - f_0^2)^2 + F^2 f_0^4 / Q^2}
\end{equation}
which may be rewritten as
\begin{equation}
P = P_z(0) \: Q \: f_0 \int_{0}^{\infty} \frac{Q \: dF}{Q^2 (F^2 - 1)^2 + F^2}
\end{equation}
The integral is of order unity: the integrand is a function that is
$\sim Q$ wide and $\sim Q$ tall, so the area of the function is
approximately one.  The integral is computed by Mathematica to be
\begin{equation}
\int_{0}^{\infty} \frac{Q \: dF}{Q^2 (F^2 - 1)^2 + F^2} = \frac{\pi}{2}
\end{equation}

We conclude that
\begin{equation}
P = \frac{\pi}{2} \: P_z (0) \: Q \: f_0
\end{equation}

\end{document}




